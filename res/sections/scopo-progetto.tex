\section{Scopo del progetto}
Il progetto realizzato ha il fine della creazione di un'applicazione semplice per la gestione del personale e dei tavoli all'interno di un ristorante/pub, dando alle diverse categorie di utente permessi differenti.

Gli utenti si dividono in: admin, cuochi e camerieri.
I tavoli possono essere di tipo pub oppure ristorante.
Le consumazioni possono avere tipo bar oppure cucina.
Ogni tavolo può avere delle consumazioni, inserite dall'utente admin o dai camerieri.
La creazione di nuovi tavoli e degli utenti è riservata all'utente admin, il quale può anche cambiare i dati personali dei camerieri, dei cuochi e di se stesso.

Ogni utente può però cambiare i proprio dati personali.

\section{Compilazione ed esecuzione}
Per compilare il progetto è necessario posizionarsi all'interno della cartella dello stesso, eseguire il comando \textbf{qmake -makefile}, eseguire il comando \textbf{make} e successivamente \textbf{./Ristorante}.
Vengono consegnati anche due file, databaseutenti.xml e databasetavoli.xml, contenenti dei dati esempio.
Nel caso in cui venissero cancellati è previsto che si crei il file denominato databaseutenti.xml con all'interno un utente amministratore con le credenziali di accesso:
\begin{itemize}
\item \textbf{username}: admin
\item \textbf{password}: admin
\end{itemize}	
